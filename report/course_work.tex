\documentclass[a4paper, fontsize=14pt]{article}
\usepackage{course_work}
\setcounter{page}{0} %в зависимости от того, какой по счёту страницей должно быть оглавление!

\begin{document}

\tableofcontents

\thispagestyle{empty}
\newpage

\section*{Введение}
\addcontentsline{toc}{section}{Введение}

% ...
Развитие вычислительной техники и вызванный этим процессом переход к более сложным (трехмерным, 
в произвольных геометрических областях) моделям в виде систем дифференциальных уравнений в 
частных производных и их дискретным аналогам на неструктурированных сетках, привел к 
необходимости решения больших разреженных систем линейных алгебраических уравнений с 
матрицами нерегулярной структуры.

Целью данной курсовой работы является изучение способов решения СЛАУ с разреженной матрицей.
Для достижения данной цели были поставлины следующие задачи:
\begin{enumerate}
    \item Изучить литературу по теме
    \item Разработать программу для решения СЛАУ с разреженной матрицей
    \item ???
    \item PROFIT
\end{enumerate}


\newpage

\section*{Заключение}
\addcontentsline{toc}{section}{Заключение}

\newpage

\addcontentsline{toc}{section}{Список литературы}

%\begin{thebibliography}{5}
%\end{thebibiliography}

\end{document}
