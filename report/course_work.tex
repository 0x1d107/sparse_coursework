\documentclass[a4paper, fontsize=14pt]{article}
\usepackage{course_work}
\setcounter{page}{0} %в зависимости от того, какой по счёту страницей должно быть оглавление!

\begin{document}

\tableofcontents

\thispagestyle{empty}
\newpage

\section*{Введение}
\addcontentsline{toc}{section}{Введение}

% ...
Развитие вычислительной техники и вызванный этим процессом пе-
реход к более сложным (трехмерным, в произвольных геометриче-
ских областях) моделям в виде систем дифференциальных уравне-
ний в частных производных и их дискретным аналогам на неструк-
турированных сетках, привел к необходимости решения больших
разреженных систем линейных алгебраических уравнений с матри-
цами нерегулярной структуры.

\newpage

\section*{Заключение}
\addcontentsline{toc}{section}{Заключение}

\newpage

\addcontentsline{toc}{section}{Список литературы}

%\begin{thebibliography}{5}
%\end{thebibiliography}

\end{document}
